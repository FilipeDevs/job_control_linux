\section{Processus en Background et Foreground}

Le terminal de contrôle conserve la notion de groupe de processus en foreground. Au sein d'une session, un seul processus peut être en foreground
à un moment donné, tous les autres groupes de processus de la session sont des groupes de processus en background. Le groupe de processus en foreground
est le seul groupe de processus qui peut lire sur le terminal de contrôle. Lorsqu'un des caractères du terminal générateur de signal est tapé sur
le terminal de contrôle, le pilote du terminal délivre le signal correspondant aux membres du groupe de processus d’en foreground.


\subsection{L’accès au terminal}

Comme expliqué précédemment, le groupe de processus en foreground est le seul groupe de processus qui peut lire sur le terminal de contrôle,
cette restriction prévient de la concurrence pour l’input du terminal. Si un processus en background essaye de lire depuis le terminal, celui-ci
se verra renvoyer un signal SIGINT, dont le comportement par défaut est de stopper le processus/job. Mais il est important de savoir que par défaut
les groupes de processus en background peuvent écrire sur le terminal, mais si le flag TOSTOP est levé pour le terminal, toute tentative d’écriture
dans le terminal se verra renvoyer le signal SIGTTOU.

Pour illustrer ce comportement utilisions le programme "terminalAccess". C'est un simple programme qui fait la meme chose que un "cat", c'est à dire
qu'il va lire depuis le terminal ce que l'utilisateur tape aprés avoir entré "Enter".

\begin{lstlisting}[caption={terminalAccess.c}, label={terminalAccess.c}]
#include <stdio.h>
#include <stdlib.h>
#include <signal.h>
#include <sys/types.h>
#include <unistd.h>

static void handleSIGTTIN(int sig) {
    printf("%ld: J'ai recu SIGTTIN\n", (long) getpid());
}
int main(void) {
	struct sigaction sa;
	sa.sa_handler = handleSIGTTIN;
	printf("Je suis le processus: PID=%ld; PPID=%ld; PGID=%ld; SID=%ld\n", (long) getpid(),
                    (long) getppid(), (long) getpgrp(), (long) getsid(0));

	// Handler pour SIGTTIN
    	if (sigaction(SIGTTIN, &sa, NULL) == -1)
        	perror("Erreur sigaction SIGINT");

	char buffer[1024];
	long n;

	printf("Je vais lire depuis le terminal...\n");
	printf("Tapez quelque chose puis Enter, j'afficherais ce que vous avez tape...\n");
	while(1) {
		n = read(STDIN_FILENO, buffer, sizeof(buffer));
		if(n < 0) {
			perror("Erreur dans read");
			break;
		}

		write(STDOUT_FILENO, buffer, n);
	}
	return 0;
}
\end{lstlisting}

Si nous l'éxécutons en foreground rien de particulier se passe, le programme marche correctement. Par contre si celui-ci s'éxécute en background, etant donné que
le processus sera en background il n'aura plus accés en stdin au terminal, le signal SIGTTIN sera retourné.
Si le signal est geré ou ignoré une erreur sera renvoyé lors de la tentative de lecture du terminal (c'est le cas dans ce programme de démonstration).

\begin{lstlisting}[language=bash]
$ ./terminalAccess &
[1] 4623
Je suis le processus: PID=4623; PPID=4595; PGID=4623; SID=4595
Je vais lire depuis le terminal...
Tapez quelque chose puis Enter, j'afficherais ce que vous avez tape...
4623: J'ai recu SIGTTIN
Erreur dans read: Interrupted system call
[1]+  Fini                    ./terminalAccess
\end{lstlisting}

Un petit cas particulier à mettre en avant est qu'il est aussi possible d'envoyer un signal pour "interdire" un processus en background d'ecrire dans le terminal. Si flag TOSTOP est levé pour
le terminal, toute tentative d'un processus en background en ecriture se verra renvoyer un signal SIGTTOU (par défaut ce signal stopppe le processus). Si le processus ignore ou gere ce signal,
il pourra quand meme ecrire sur le terminal.

\begin{lstlisting}[language=bash]
$ stty tostop
$ date &
[1] 4725
[1]+  Stopped               date
\end{lstlisting}

\subsection{Fonctions \texttt{tcgetpgrp()} et \texttt{tcsetpgrp()}}

Les fonctions \texttt{tcgetpgrp()} et \texttt{tcsetpgrp()} jouent un rôle important dans la manipulation des groupes de processus associés à un terminal, notamment dans les shells de contrôle de jobs. Explorons ces fonctionsl :

\begin{itemize}
    \item \texttt{pid\_t tcgetpgrp(int fd)} : La fonction \texttt{tcgetpgrp()} est utilisée pour récupérer le groupe de processus en premier plan (foreground) associé à un terminal spécifique. En tant qu'argument,
 elle prend le descripteur de fichier \texttt{fd} qui est lié au terminal en question. La fonction retourne le PGID (Process Group ID) du groupe de processus en foreground associé au terminal.

    \item \texttt{int tcsetpgrp(int fd, pid\_t pgid)} : La fonction \texttt{tcsetpgrp()} est utilisée pour définir le groupe de processus en foreground associé à un terminal spécifique. Elle prend en argument le descripteur de fichier \texttt{fd} associé
 au terminal et le PGID du groupe de processus que l'on souhaite définir en foreground. Le retour de la fonction indique le succès ou l'échec de l'opération.
\end{itemize}

Remarques :
\begin{itemize}
    \item Sur de nombreuses implémentations Unix, y compris Linux, ces deux fonctions sont souvent implémentées en utilisant des opérations ioctl (Entrées/Sorties Contrôlées) non standardisées, telles que \texttt{TIOCGPGRP} et \texttt{TIOCSPGRP}.
    \item Ces fonctions sont particulièrement utiles dans le contexte des shells de contrôle de jobs, permettant aux shells de gérer les processus en foreground et en background associés à un terminal.
\end{itemize}

Une démonstration illustrant une de ces fonctions sera vu plus tard.



