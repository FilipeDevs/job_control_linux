\section{Terminales de contrôle}
Un terminal de contrôle est un terminal spécial associé à une session de processus. Il sert de point de communication principal entre une session et l'utilisateur, permettant aux processus de cette session d'interagir avec l'utilisateur via le terminal. Tous les processus d’une session peuvent avoir un seul et unique terminal de contrôle.

\subsection{Association à un terminal de contrôle}

Comme expliqué précedemment, lorsqu'une nouvelle session est créée, le processus leader de cette session n'a pas encore de connexion au terminal de contrôle. En d'autres termes, à sa création, une session ne dispose pas encore d'un terminal de contrôle associé.
\newline
Pour qu'une session puisse établir une connexion avec un terminal de contrôle, le processus leader de cette session doit ouvrir un nouveau terminal de contrôle. Il est essentiel de noter que ce terminal ne doit en aucun cas être déjà associé à une autre session. 
\newline
Lorsque le leader de la session réussit à établir cette connexion avec le terminal, il devient ce que l'on appelle le "processus contrôleur". Les détails sur les implications de ce rôle seront expliqués dans la suite. 
\newline
Tous les processus dans une session héritent de l’accès du terminal de contrôle. 
Habituellement, les nouvelles sessions sont créées par le programme de login du système et le leader de la session est le processus exécutant le shell de login de l'utilisateur.
