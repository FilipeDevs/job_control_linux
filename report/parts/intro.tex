\section{Introduction}
Le contrôle des jobs dans un système d'exploitation Linux permet l'exécution simultanée de plusieurs commandes avec un contrôle précis. Cette fonctionnalité permet d'assigner un job en "foreground" pendant que d'autres s'exécutent en "background", simplifiant ainsi la gestion des tâches. On peut arrêter, déplacer et surveiller les jobs en cours, offrant une solution pratique pour effectuer diverses opérations sans ouvrir de nouveaux terminaux.
\newline
Cette recherche a été initiée pour approfondir la compréhension des mécanismes régissant le contrôle des jobs dans l'environnement Linux. Le rapport est structuré en plusieurs sections, chacune se concentrant sur un aspect particulier du contrôle des jobs. L'objectif principal est de présenter une vue d'ensemble exhaustive de cette fonctionnalité, en débutant par ses fondements pour ensuite explorer des concepts plus avancés.
\newline
