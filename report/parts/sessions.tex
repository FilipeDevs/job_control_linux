\section{Sessions}
Une session est un regroupement de plusieurs groupes de processus. La participation d'un processus à une session est déterminée par son identifiant de session (SID). Le leader de session est le processus responsable de la création d'une nouvelle session et dont sont identifiant (PID) devient l'identifiant de session (SID) de cette session (généralement le shell). Lorsqu'un nouveau processus est créé, il hérite automatiquement de l'identifiant de session (SID) de son processus parent.

\subsection{L’appel système \textit{setsid()} }
L’appel système \textit{setsid()} permet de créer une nouvelle session et définir le processus appelant comme le leader de cette session. Le PGID et SID deviennent le même que le PID du processus appelant.
\newline
\begin{lstlisting}
#include <unistd.h>

pid_t setsid(void);
\end{lstlisting}
Il est important de noter que, d'une part, un leader de groupe de processus existant ne peut pas créer une nouvelle session. D'autre part, lorsqu'un processus fait l'appel à \textit{setsid()}, il perd toute connexion préexistante au terminal de contrôle. 
\newline
Le code ci-dessous, illustre la création d'une nouvelle session, pour démontrer que cette nouvelle session n'a plus d'accés au terminal de contrôle, le programme essaye d'ouvrir
le terminal de contrôle associé à la session. Une éxécution de ce programme entraine un résultat similaire à celui-ci  :

\begin{lstlisting}
$ ./creationSession
Avant nouvelle session : PID=1023, PGID=1022, SID=944
Apres nouvelle session : PID=1023, PGID=1023, SID=1023
open /dev/tty : No such device or address
\end{lstlisting}



\begin{lstlisting}[caption={creationSession.c}, label={creationSession.c}]
#include <unistd.h>
#include <fcntl.h>
#include <stdio.h>
#include <errno.h>
#include <stdlib.h>  

int main(void) {

    if (fork() != 0) // Exit du parent
        exit(0);

    printf("Avant nouvelle session : PID=%ld, PGID=%ld, SID=%ld\n", (long) getpid(),
        (long) getpgrp(), (long) getsid(0));

    if (setsid() == -1)
        perror("setsid");
        exit(-1);

    printf("Avant nouvelle session : PID=%ld, PGID=%ld, SID=%ld\n", (long) getpid(),
        (long) getpgrp(), (long) getsid(0));

    if (open("/dev/tty", O_RDWR) == -1) // Ouvrir terminal de controle associe
        perror("open /dev/tty");
        exit(-1);

    exit(0);
}

\end{lstlisting}







